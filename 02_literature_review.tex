\section{Main Body}~\label{sec:literature}
There are several trajectory analysis techniques built in the fields of Geographic Information Science and Data Mining. These strategies may search for activity dynamics like flocking, leadership, commuting, and encounter, or they may be targeted at answering simple questions like nearest neighbour or range queries.e we mainly focus on queries that are used for aggregate statistics. This queries are typically measurable, and thus they can be defined as functions
on the domain of all spatio-temporal databases ranging over a metric space.
Spatial-temporal range queries, such as Sometime Definitely Inside and Always Definitely Inside, are a well-known form of observable query in trajectory analysis. Sometime Definitely Inside is valid if and only if trajectory T occurs at a time when it is inside area R. Always Definitely Inside is accurate if and only if trajectory T is always inside area R.\\
We compared our method to the generalisation-based and permutation-based approaches suggested in. The distance threshold used in the generalisation-based approach requires the Log-cost distance calculation to discard outlier locations. Since we are only considering noiseless synthetic data in this section, we have set such a distance threshold to its maximum value. Instead, the permutation-based approach discards outlier positions during the obfuscation step by taking both a distance and a time threshold into account. For any cluster size and time period, the system outperforms the methods suggested in. When the level of privacy given improves, so does the change in utility. For k = 2, our method performs marginally better than the generalisation-based solution, while for k = 4, 8, our method performs substantially better. This ensures that our method effectively groups and anonymizes trajectories.
