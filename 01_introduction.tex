%========= Introduction
\section{Introduction}~\label{sec:introduction}
Location based devices, such as cell phones and GPS receivers, allow businesses and governments to quickly gather massive amounts of data about people's movements.Analyzing and mining this type of data, also known as trajectories, may uncover previously unknown patterns and expertise that can be used in traffic, sustainable mobility management, urban planning, and supply chain management. This allows capital to be optimised and corporate and government decisions to be solid and well-founded. In this way, all businesses and individuals benefit directly from the publishing and study of trajectory datasets.\\
De-identification, which involves deleting distinguishing characteristics from people, is a possible method for achieving anonymity. However, this is often inadequate to protect privacy due to the presence of other types of attributes known as quasi-identifiers, which are non-identifying attributes that, when combined with external knowledge, will uniquely identify the person behind a database. Unfortunately, in the case of spatiotemporal results, any position can be considered a quasi-identifier [26].
As a result, simply knowing where a person has been might be enough to classify his path in a database.\\
Here suggest a distance metric for trajectories that is particularly suitable for clustering and obfuscation. The distance is loosely based on the Fr'echet distance, but it is computationally efficient. The novel construction has several advantages: (i)t can work with non-overlapping trajectories, (ii) it outputs a number of corresponding points in addition to a distance value, which is later used in the obfuscation method, and (iii) it takes into account the form of the trajectories due to the existence of the Fr'echet distance. We use the proposed distance metric as the foundation of a trajectory anonymisation strategy that produces datasets that fulfil k-anonymity independently of adversary awareness.

